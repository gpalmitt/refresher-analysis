\documentclass[10pt]{beamer}
\usepackage[utf8]{inputenc}
%\usefonttheme{structuresmallcapsserif}
%\usetheme{Madrid}
\reversemarginpar
\usetheme{Madrid}
\usecolortheme{whale}
\usepackage{amsthm}
\usepackage{amssymb}
\usepackage{xcolor}
\usepackage{eurosym}
\usepackage{pifont}
\usepackage{fancybox}
\usepackage{multicol}
\usepackage{amsmath, mathtools}
\usepackage{amssymb}
\usepackage{graphicx}
\usepackage{multirow}
\usepackage{color}
\usepackage{graphicx}
\usepackage{booktabs}
\usepackage{array}
\usepackage{pifont, amsfonts, eufrak}
\usepackage{multicol}
\usepackage{caption}
\usepackage{amssymb,latexsym } 
\usepackage{rotating}
\usepackage{bbm}
\usepackage{cases}
\usepackage{pifont}
%\usepackage{mathabx}
\usepackage{dsfont}
%\usepackage{fourier, heuristica}
\usepackage{array, booktabs}
%\DeclareCaptionFont{blue}{\color{LightSteelBlue3}}

\newcommand{\foo}{\color{blue}\makebox[0pt]{\textbullet}\hskip-0.5pt\vrule width 1pt\hspace{\labelsep}}

\usepackage{etoolbox}

\newcommand\Pitem{%
  \addtocounter{enumi}{-1}%
  \renewcommand\theenumi{\arabic{enumi}'}%
  \item%
  \renewcommand\theenumi{\arabic{enumi}}%
}

\definecolor{or}{rgb}{1.0, 0.4, 0.0}
\definecolor{greng}{rgb}{0.0, 0.62, 0.38}
\definecolor{gro}{rgb}{0.77, 0.76, 0.75}
\definecolor{blo}{rgb}{0.15, 0.38, 0.61}
\definecolor{roud}{rgb}{0.81, 0.06, 0.13}

%\usepackage{lmodern}
%\newcommand\itemshape[1]{%
%\setbeamertemplate{itemize item}[#1]{%
%\usebeamertemplate{itemize item}
%}

\def\X#1{%
        %#1%
        %\textcircled{#1}%
        \raisebox{.9pt}{\textcircled{\raisebox{-.9pt}{#1}}}%
        %\ding{\numexpr171+#1\relax}%
}


\usepackage{tikz}
\usepackage{tikz-cd}
\usetikzlibrary{arrows}
\usetikzlibrary{positioning}
\usepackage{caption}
\usetikzlibrary{shapes,arrows}
\usepackage[all]{xy}

\setbeamertemplate{navigation symbols}{}

\DeclareMathOperator{\spn}{span}
\newcommand{\R}{\mathbb{R}}
\newcommand{\C}{\mathbb{C}}
\newcommand{\N}{\mathbb{N}}
\newcommand{\Z}{\mathbb{Z}}
\newcommand{\E}{\mathbb{E}}
\newcommand{\F}{\mathbb{F}}
\newcommand{\HH}{\mathbb{H}}
\newcommand{\VV}{\mathbb{V}}
\newcommand{\cc}{\mathbf{c}}
\newcommand{\CC}{\mathbf{C}}

\newcommand{\U}{\mathcal{U}}
\newcommand{\DD}{\mathcal{D}}
\newcommand{\g}{\mathfrak{g}}
\newcommand{\kk}{\mathfrak{k}}
\newcommand{\aL}{\mathfrak{a}}
\newcommand{\End}{\text{End}}
\newcommand{\Hom}{\text{Hom}}
\newcommand{\Hol}{\text{Hol}}
\newcommand{\Pol}{\text{Pol}}
\newcommand{\opp}{\text{opp}}
\newcommand{\Ad}{\text{Ad}}
\newcommand{\Ker}{\text{Ker}}
\newcommand{\Impart}{\text{Im}}
\newcommand{\Repart}{\text{Re}}
\newcommand{\supp}{\text{supp}}
\newcommand{\Mat}{\text{Mat}}
\newcommand{\Tr}{\text{Tr}}
\newcommand{\Id}{\text{Id}}
\newcommand{\rk}{\text{rk}}
\newcommand{\diag}{\text{diag}}
\newcommand*{\QEDA}{\null\nobreak\hfill\ensuremath{\square}}%


\setbeamertemplate{theorems}[numbered] 

\AtBeginSection[]
{
  \begin{frame}
    \tableofcontents[currentsection]
  \end{frame}
}
\logo{\includegraphics[height=1cm]{dmath}}
\title[Refresher Courses in Analysis] %optional
{Chapter 3: Basic of complex analysis for functions in one variable}

\subtitle{Refresher Courses in Analysis}
\author[Guendalina \textsc{Palmirotta}]{Guendalina \textsc{Palmirotta}}
\institute[]{University of Luxembourg, Department of Mathematics}
\date[]{\today}
\newcommand{\nologo}{\setbeamertemplate{logo}{}}
\begin{document}


\begin{frame}
\titlepage
\end{frame}

{\nologo


%\begin{frame}
%\frametitle{Contents}
%\tableofcontents
%\end{frame}

\section{3.1. Holomorphic functions and Cauchy-Riemann equations}

\begin{frame}
\frametitle{Complex differentiable and holomorphic functions}
\begin{alertblock}{Definition 1}
Let $U \subset \C$ be open and $z_0 \in U$.
\begin{itemize}
\item[(i)] $f:U \rightarrow \C$ is called \textbf{complex differentiable} at $z_0$ if
\vspace{0.5cm}
\item[(ii)] $f$ is called \textbf{holomorphic} on $U$ if it is complex differentiable at all $z_0 \in U$.
\item[(iii)] Hol$(U):=\{f: U \rightarrow \C \;| \; f$ holomorphic on $U\}$ denotes the space of holomorphic functions on $U$.
\end{itemize}
\end{alertblock}
\vspace{0.2cm}
\underline{Note}: that the existence of the above limit is a much stronger cond. that in the real case.
In other words:
\vspace{1cm}
\end{frame}

\begin{frame}
\frametitle{Examples}
\begin{itemize}
\item[1)] (polynomials)\;\\
\vspace{1cm}
\underline{Fact} (Lionville) Every bounded entire function (i.e. $f \in$ Hol$(\C)$) is constant.\\
\vspace{0.2cm}
\item[2)] (power series)\;\\
\vspace{1.5cm}
$\Rightarrow$ Power series (and polynomials) are implicitly often complex differentiable.\\
$\Rightarrow$ Every hol. fct is locally given by a power series, i.e. it is complex analytic.
\vspace{0.2cm}
\item[3)]\;\\
\vspace{1.5cm}
\end{itemize}
\end{frame}

\begin{frame}
\frametitle{Examples}
\begin{itemize}
\item[4)] Let $f,g\in$ Hol$(U), U \subset \C$ open.\\
\vspace{3cm}
\item[5)] Let $U,V \subset \C$ open, $f\in$ Hol$(V)$ and $g:U \rightarrow V \subset \C$.\\
\;\\
\vspace{1.5cm}
\end{itemize}
\end{frame}

\begin{frame}
\frametitle{Cauchy-Riemann DE}
 Let $U \subset \C$ open and $f: U \rightarrow \C$ be a function.\\
\vspace{1cm}
\underline{Aim}: Compare the notion of being hol.  with differentiability of $f$ as of 2-real variables.\\
\vspace{0.2cm}
\begin{block}{Proposition 1}
The following conds. are equivalent:
\begin{itemize}
\item[(i)] $f$ is hol. on $U$.
\item[(ii)] $f$ is ``real differentiable'' on $U$ and $\forall z \in U$, the \underline{real} linear map $df(z):\C \rightarrow \C$ is given by multipilcation with a complex nb $a_z \in \C$, i.e. $df(z)$ is complex linear.
\item[(iii)] $f$ is real diff. and $\forall z \in U$\\
\vspace{1cm}
\end{itemize}
\end{block}
\end{frame}

\begin{frame}
\frametitle{Remarks}
\begin{itemize}
\item[(i)] The C.-R. eq. are often often decomposed into real and imaginary part:\\
\vspace{2cm}
By writing $df(z)$ as a $2\times 2$ matrix,  we have\\
\vspace{2cm}
$(**)$ corresponds to the fact that a matrix $\begin{pmatrix} a\; b \\ c\; d\end{pmatrix}$ is complex linear $\iff$ $a=d$ and $b=-c$.
\end{itemize}
\end{frame}

\begin{frame}
\frametitle{Remarks}
\begin{itemize}
\item[(ii)] We can introduce a new complex basis of partial derivatives\\
\vspace{2cm}
Then $(*)$ is equivalent to \\
\vspace{1cm}
Moreover, if $f$ is hol.,  then\\
\vspace{1cm}
So, the notation is compatible with the notation
\\
\vspace{1cm}
\end{itemize}
\end{frame}

\begin{frame}
\frametitle{Remarks}
\begin{itemize}
\item[(iii)] A $\mathcal{C}^2$-fct on $U \subset \R^2 \cong \C$ is called \textbf{harmonic} if\\
\vspace{1cm}
Assume that $f$ is hol., then\\
But
\\
\vspace{1cm}
So hol.  fcts are harmonic!
\end{itemize}
\end{frame}

\begin{frame}
\frametitle{(Locally) biholomorphic functions}
$\Rightarrow$ Recall the Inverse function theorem (Thm.~3) in Chap.~1.\\
\vspace{0.2cm}
\underline{Holomorphic version of diffeomorphism}
\vspace{0.2cm}
\begin{alertblock}{Definition 2}
\begin{itemize}
\item Let $U,V \subset \C$ be open. $f:U \rightarrow V$ is called \textbf{bihol.} if it is
\begin{itemize}
\item[(i)] holomorphic,
\item[(ii)] bijective, and
\item[(iii)] $f^{-1}:U \rightarrow V$ is hol.
\end{itemize}
\item $f:U \rightarrow \C$ is called \textbf{locally bihol.} at $z\in U$ if
\begin{itemize}
\item[(i)] $\exists$ an open neigh. $U_z$ of $z\in U$,
\item[(ii)] $f(U_z):=U_z$ is open,  and
\item[(iii)] $f|_{U_z}:U_2 \rightarrow V_z$ is bihol.
\end{itemize}
\end{itemize}
\end{alertblock}
\vspace{0.2cm}
\begin{block}{Proposition 2}
\begin{itemize}
\item[(a)] $f$ is locally bihol. at $z\in U \iff f'(z) \neq 0$ for $f\in$ Hol$(U), z \in U$.
\item[(b)] $f\in$ Hol$(U)$ injective $\iff f'(z) \neq 0 \forall z \in U$ and $f:U \rightarrow V:=f(U)$ is bihol.
\end{itemize}
\end{block}
\end{frame}

\section{3.2.  Identity theorem and Maximum principle}

\begin{frame}
\frametitle{Domain and maximum principle}
\begin{alertblock}{Definition 3 (domain)}
A subset $U\subset \C$ is called a \textbf{domain} if it is open and connected.
\end{alertblock}
\vspace{0.2cm}
\begin{block}{Proposition 3}
\begin{itemize}
\item[(a)] $U\subset \C$ be a domain and $f\in$ Hol$(U)$ non-constant $\Rightarrow$ $f(U) \subset \C$ is also a domain.
\item[(b)] (``Maximum principle") $U\subset \C$ be a domain and $f\in$ Hol$(U)$ non-constant $\Rightarrow$ the fct\\
\vspace{0.5cm}
has no (global) maximum on $U$.
\end{itemize}
\end{block}
\underline{Remarks}:
\begin{itemize}
\item By  making $U$ smaller it is also clear that $|f|$ has no local max. on $U$.
\item \textcolor{blue}{(b)} is often applied in the following way:
\\
\vspace{1cm}
\end{itemize}
\end{frame}

\begin{frame}
\frametitle{Accumulation point}
\begin{alertblock}{Definition 4 (accumulation point)}
Let $(X,d)$ be a metric space.
\begin{itemize}
\item[(a)] A point $x_0\in X$ is called \textbf{isolated} in $X$ if $\exists r>0$ s.t.
$$B(x_0,r)=\{x_0\}.$$
In other words, $\{x_0\} \subset X$ is open.
\item[(b)] Let $A \subset X$ be a subset. Then $x_0 \in X$ is called an \textbf{accumulation point} of $A$ in $X$ if there is a sequence $(a_n)_{n\in \N}$ in $A\backslash\{x_0\}$ s.t.
$$a_n \rightarrow x_0.$$
\end{itemize}
\end{alertblock}
\vspace{0.2cm}
\underline{Remark}: $x_0\in X$ is an accumulation point of $A$ if $x_0 \in \overline{A}$ and $x_0$ is \textbf{not} an isolated point of $\overline{A}$. The set of accumulation points of $A$ is closed.
\end{frame}

\begin{frame}
\frametitle{Zeros of hol. fcts}
\begin{block}{Proposition 4 (zeros of hol. fcts)}
Let $U \subset \C$ be a domain,  $f\in$ Hol$(U)\backslash\{0\}$ and $Z(f):=\{z_0\in U | f(z_0)=0\}.$\\
\textbf{Then}, $Z(f)$ has no accumulation point in $U$ (equiv.: every $z_0\in Z(f)$ is isolated in $Z(f)$) and $Z(f)$ is at most countable.\\
\vspace{0.2cm}
For every $z_0\in Z(f)\; \exists m \in \N$ and $g\in$ Hol$(U)$ with $g(z_0)\neq 0$ s.t.\\
\vspace{2cm}
\end{block}
\vspace{0.2cm}
\underline{Remark/Example}: $Z(f)$ may have accumulation point outside $U$ in $\C$. \\
For instance: $f\in \C\backslash \{0\} \rightarrow \C$ by $f(z)=\sin(\frac{\pi}{z})$,  we have\\
\vspace{1cm}
Thus, $0$ is an accumulation point but $0\in \C\backslash \{0\}$.
\end{frame}

\begin{frame}
\frametitle{Identity theorem}
\begin{block}{Corollary 1 (``Identity theorem")}
Let $U \subset \C$ be adomain and $f,g \in$ Hol$(\C)$.
Define
$$\;$$
If $Z$ has an accumulation point in $U$, then $f=g$.
\end{block}
\vspace{0.2cm}
\underline{Remarks}: Identity Thm. says in particular that:
\begin{itemize}
\item $f\in$ Hol$(U), U$ domain, is uniquely determined by
$$f|_{B(z_0,\epsilon)}, z_0\in U, \epsilon>0 \text{ arbitrary.}$$
\item $f\in$ Hol$(U), U$ domain, $U \cap \R \neq \emptyset$.\\
Then $U \cap \R \subset \R$ is open, non-empty, and it has an accumulation point. Hence $f$ is uniquely determined by  $f|_{U \cap \R}$.
\end{itemize}
\end{frame}

\section{3.3.  Isolated singularities and meromorphic function}

\begin{frame}\frametitle{Isolated singularities}
\underline{Motivation}: $U \subset \C$ open, $f\in$ Hol$(U)$.\\
If $z_0 \neq \C$, then $f(z_0)$ is not defined!\\
$\Rightarrow$ We may think of $z_0$ as a ``singularity'' of $U$.
\vspace{0.2cm}
\begin{alertblock}{Definition 5 (isolated singularity)}
Let $U \subset \C$ be open and $f\in$ Hol$(U)$.\\
A point $z_0\in \C\backslash U$ is called an \textbf{isolated singularity} of $f$, if $z_0$ is an isolated point of $ \C\backslash U$, i.e.\\
\vspace{1cm}
\end{alertblock}
\vspace{0.2cm}
\underline{Example}: \vspace{2cm}
\end{frame}

\begin{frame}\frametitle{Types of isolated singularities}
There are 3 types of isolated singularities:
\vspace{0.2cm}
\begin{alertblock}{Definition 6}
Let $U \subset \C$ be open, $f\in$ Hol$(U)$ and $z_0\in U$ isolated sing. of $f$.\\
\begin{itemize}
\item[(i)] $z_0$ is called \textbf{removable} if $\tilde{f}\in$ Hol$(U \cup \{z_0\})$ s.t. $\tilde{f}|_U=f$.
\item[(ii)] We say that $f$ has a \textbf{pole} at $z_0$, if $z_0$ is not removable but
$$\;$$
has a removable singularity at $z_0$ for some $m>0$. The smallest $m$ with this property is called the \textbf{order} of the pole of $f$ at $z_0$.
\item[(iii)] $z_0$ is called \textbf{essential} singularity of $f$ if it is neither removable nor a pole.
\end{itemize}
\end{alertblock}
\end{frame}

\begin{frame}\frametitle{Types of isolated singularities}
\underline{Examples}:\\
 \vspace{0.2cm}
\begin{itemize}
\item[(1)] \;\\ \vspace{1cm}
\item[(2)] \;\\ \vspace{4cm}
\end{itemize}
\end{frame}

\begin{frame}\frametitle{Meromorphic function}
\begin{alertblock}{Definition 7 (meromorphic function)}
Let $U \subset \C$ be open.\\
A \textbf{meromorphic function} on $U$ is a holomorphic fct $f: U_0 \rightarrow \C$ on an open subset $U_0 \subset U$ s.t.
$$\;$$
Let $M(U)$ be the set of meromorphic fcts in $U$.
\end{alertblock}
\vspace{0.1cm}
\underline{Remarks}:
\begin{itemize}
\item $S$ consists of isolated points, therefore it is at most countable.
\item Strictly speaking a meromorphic fct $f$ on $U$ is \textbf{not} a fct on $U$, only a fct on $U_0=U \backslash S$. Moreover, one can think of $f$ as a fct\\
\vspace{0.5cm}
%$$f:U \rightarrow \C \cap \{\infty\} \text{ ``Riemann sphere''}$$
by setting $f(z)=\infty, z \in S$.
\item If $U \subset \C$ is a domain, then $(M(U),+, \cdot)$ is a field with multiplicative unit given by the constant fct 1. In particular, for $f,g \in M(U), g \neq 0$, we have\\
\vspace{0.5cm}
\end{itemize}
\end{frame}

%\begin{frame}\frametitle{Laurent series}
%\begin{alertblock}{Definition 8 (Laurent series)}
%Let $z_0\in \C$.
%\begin{itemize}
%\item[(i)] A \textbf{Laurent series} with center $z_0$ is a series of the form\\
%\vspace{0.7cm}
%its principal part (or singular part) is given by\\
%and its hol. part (or regular part):\\
%\item[(ii)] $L(z)$ converges at $z\in \C\backslash\{0\}$ if both parts converge.\\
%Moreover, in the convergent case, we denote \\
%\vspace{0.5cm}
%\item[(iii)] If $R$ is the radius of convergence part of the hol. part and $\frac{1}{r} \in [0,\infty)$  is the radius of convergence part of the principal part, then\\
%\vspace{0.7cm}
%\end{itemize}
%\end{alertblock}
%\end{frame}

%\begin{frame}\frametitle{Laurent series  expansion}
%\underline{Remarks}:
%\begin{itemize}
%\item $L(z)$ converges inside the ``annules'' 
%\item $L(z)$ diverges on $\C \backslash \overline{A_{r,R}(z_0)}$.
%\end{itemize}
%\vspace{0.2cm}
%\begin{block}{Proposition 5  (Laurent series expansion)}
%Let $0 \leq r < R \leq \infty, z_0 \in \C, f\in$ Hol$(A_{r,R}(z_0))$.\\
%\textbf{Then}, $f$ has a unique Laurent series expansion corresp. in $A_{r,R}(z_0)$:\\
%\vspace{1cm}
%We have \\
%\vspace{0.7cm}
%\end{block}
%\vspace{0.2cm}
%\underline{Main application}: $r=0, A_{0,R}(z_0)=B(z_0,R)\backslash \{z_0\}$.\\
%$f\in$ Hol$(A_{0,R}(z_0))$ has an isolated sing. at $z_0$.
%\end{frame}

%\begin{frame}\frametitle{Laurent series  expansion}
%\begin{block}{Corollary 2}
%Let $U \subset \C$ be open,  $f\in$ Hol$(U)$ and $z_0 \in \C\backslash U$ isolated sing.\\
%\textbf{Then}, $\exists R>0$ s.t.
%\begin{itemize}
%\item[(i)] $B(z_0,R)\backslash \{z_0\} \subset U$ and
%\item[(ii)] $f$ has a unique Laurent expansion converging on $B(z_0,R)\backslash \{z_0\}$:\\
%\vspace{0.5cm}
%The isolated sing. $z_0$ is:
%\begin{itemize}
%\item[(a)] \textbf{removable} $\iff a_n=0\; \forall n <0$ (principal part vanishes)
%\item[(b)] a \textbf{pole} of order $m>0 \iff a_{-m} \neq 0$ and $a_n=0 \;\forall n <-m$
%\item[(c)] \textbf{essential} $\iff$ there are infinitely many $n<0$ with $a_n\neq 0$ (principal part is finite).
%\end{itemize}
%\end{itemize}
%\end{block}
%\vspace{0.2cm}
%\underline{Example}:\; \\
%\vspace{2cm}
%\end{frame}

\section{3.4.  Complex path integration and Residue theorem}

\begin{frame} \frametitle{Residue}
\underline{Recall}: For $f\in$ Hol$(U)$,  Laurent series with center $z_0\in U\backslash \C$ is a series of the form
$f(z)=\sum_{n=-\infty}^\infty a_n(z-z_0)^n,$
where $$a_n=\frac{1}{2\pi i} \int_{|z-z_0|=\epsilon} \frac{f(z)}{(z-z_0)^{n+1}} \; \text{d}z.$$
\underline{Focus}: on the coefficient $a_{-1}$.
\begin{alertblock}{Definition 8 (residue)}
Let $f\in$ Hol$(U), z_0\in U\backslash \C$ an isolated sing. Choose $\epsilon>0$ sufficiently small s.t. $\overline{B}(z_0,\epsilon)\backslash \{z_0\} \subset U$. \\
Then, the \textbf{residue} $Res_{z_0}(f)\in \C$ of $f$ at $z_0$ is defined as\\
\vspace{1.5cm}
\end{alertblock}
\underline{Example}: $Res_{z_0}(e^{1/z})=1.$
\end{frame}

\begin{frame} \frametitle{Ways to compute $Res_{z_0}(f)$}
\begin{itemize}
\item[(i)] $z_0$ removable sing.: $Res_{z_0}(f)=0$.\\
\vspace{0.2cm}
\item[(ii)] $z_0$ a pole of order 1 (``principale pole'') $\Rightarrow  Res_{z_0}(f)=\lim_{z \rightarrow z_0}(z-z_0)f(z).$\\
\vspace{0.2cm}
\vspace{0.2cm}
\item[(iii)] $z_0$ a pole of order at most $m$: $Res_{z_0}(f)=\frac{1}{(m-1)!}\lim_{z \rightarrow z_0}\big(\frac{d}{dz}\big)^{m-1}(z-z_0)^m f(z)$.\\
\vspace{0.2cm}
\item[(iv)] $f$ has a primitive $F\in$ Hol$(B(z_0,\C)\backslash\{z_0\}) \iff Res_{z_0}(f)=0$.\\
\vspace{0.2cm}
\item[(v)] $f,g$ hol. around $z_0$ and $g$ has a zero of order 1 at $z_0 \Rightarrow  Res_{z_0}(f/g)=\frac{f(z_0)}{g(z_0)}.$

\end{itemize}
\end{frame}

\begin{frame} \frametitle{Integration along paths and cycle}
\begin{alertblock}{Definition 9 (path and cycle)}
Let $U\subset \C$ be open.
\begin{itemize}
\item[(a)] A \textbf{curve} in $U$ is a cont. map $\gamma:[a,b] \rightarrow U$, $[a,b] \subseteq \R$ closed bounded intervall.
\item[(b)] A \textbf{closed curve} is a curve with $\gamma(a)=\gamma(b)$. 
\item[(c)] A \textbf{path} is a piecewise cont. diff. curve $\gamma$, i.e.\\
\vspace{0.7cm}
\item[(d)] A \textbf{closed path} is a path which is a closed curve.
\item[(e)] A \textbf{cycle} $\gamma$ in $U$ is a formal finite linear combination of closed paths in $U$ with integer coeff:\\
\vspace{0.7cm}
where $\gamma_i:[a_i,b_i] \rightarrow \C$ closed path, $\lambda_i\in \Z$.
\end{itemize}
\end{alertblock}
\underline{Notation}: $\gamma^*:=\{\gamma(t)|t \in [a,b]\} \subset U$ denotes the image of the curve $\gamma$.
\end{frame}

\begin{frame} \frametitle{Integration along paths and cycle}
\begin{alertblock}{Definition 10 (integration along paths and cycle)}
\begin{itemize}
\item[(i)] Let $\gamma:[a,b] \rightarrow U \subset U$ be a path and $f:\gamma^*\rightarrow \C$ cont.\\
We define\\
\vspace{1cm}
\item[(ii)]Let $\gamma$ be a cycle. For any cont. $f:U\rightarrow \C$ we define\\
\vspace{1cm}
\end{itemize}
\end{alertblock}
\underline{Remarks}:
\begin{itemize}
\item Integral can be taken as Riemann or Lebesgue, more in the sense of an oriented integral. 
%\item $\gamma'(t)$ may not be well-defined at the ends of the pieces $s_1, \dots,s_{n-1}.$ These are only finitely many points.  Therefore the integral is well-defined, namely as
%$$\sum_{i=0}^{n-1} \int_{s_i}^{s_{i+1}} f(\gamma(t))\cdot \gamma'(t) \; \text{d}t.$$
\item The length of $\gamma$ is defined as $l(\gamma)=\int_{a}^b |\gamma'(t)| \; \text{d}t \in [0, \infty).$
\end{itemize}
\end{frame}


\begin{frame} \frametitle{Examples}
Here $z_0\in \C$ is fixed.\\
\vspace{0.2cm}
\begin{itemize}
\item[(1)] \textbf{Constant path}: $\gamma:[a,b] \rightarrow \C, \gamma(t)=z_0.$\\
We have that $\gamma'(t)=0$ and thus\\
\vspace{1cm}
\item[(2)] \textbf{Circles} (positively oriented): 
Given $r>0$, define $\gamma:[0,2\pi] \rightarrow \C, \gamma(t)=z_0+re^{it}$ closed path.
We have that $\gamma'(t)=ire^{it}$ and thus\
\vspace{1cm}
\end{itemize}
\end{frame}

\begin{frame} \frametitle{Winding number}
\begin{alertblock}{Definition 11 (winding number)}
Let $\gamma:[a,b] \rightarrow \C$ be a closed curve and $z\in \C \backslash \gamma^*$.\\
We define the \textbf{winding number} of $\gamma$ around $z$ by
$$n_\gamma(z):=\frac{1}{2\pi i} \int_\gamma \frac{1}{\xi-z} \; \text{d}\xi \in \Z.$$
In particular, for $z\in \C\backslash \gamma^*$ with $\gamma$ a cycle in $U$, the winding number is
$$n_\gamma(z):=\sum_{i=1}^r \lambda_i n_{\gamma_i}(z)\in \Z$$
is well-defined.
\end{alertblock}
\underline{Examples}:\\
\vspace{2cm}
\end{frame}

\begin{frame} \frametitle{Residue theorem}
\begin{block}{Theorem 1 (residue theorem)}
Let $U\subset \C$ be open, $f$ hol. on $U$ up to isolated sing., i.e. $\exists U_0 \subset U$ open, $f\in$ Hol$(U_0)$ and every $a\in S:=U\backslash U_0$ is an isolated sing. of $f$.\\
Let $\gamma$ be a cycle in $U_0=U \backslash S$ s.t.  $n_\gamma(z)=0, \forall z\in \C\backslash U$, i.e. $\gamma$ only around isolated sings. of $f$.\\
\textbf{Then}, the set $S_1=\{a \in S|n_\gamma(a) \neq 0\}$ is finite and we have the residue formula:\\
\vspace{1cm}
\end{block}
\underline{Example} Compute $\int_0^\infty \frac{1}{1+x^n} \; \text{d}x.$\\
Let $f(z)=\frac{1}{1+z^n}, n \geq 2$.  Observe that $1+z^n=0 \iff z=\theta^l, l$ odd with $\theta=e^{\pi i/n}$.
Then $S_1=\{0\}$ and $n_\gamma(\theta)=1$ indep. of $r>1$.\\
Hence $\int_{\gamma_r} \frac{1}{1+z^n} \; \text{d}z= 2\pi i \;Res_{\theta}(\frac{1}{1+z^n})$ with
$Res_{\theta}(\frac{1}{1+z^n})=\frac{1}{(1+z^n)'}\Big|_\theta=[\dots]=\frac{-e^{\pi i/n}}{n}.$
$$\Rightarrow \int_0^\infty \frac{1}{1+x^n}=\frac{2\pi i}{n} \cdot \frac{e^{-\pi i/n}}{1-e^{2\pi i/n}}=\frac{\pi}{n}\cdot \frac{1}{\sin(\pi/n)}.$$
\end{frame}


\section{3.5.  General applications to computation of real integrals}

\begin{frame} \frametitle{Application1}
\underline{Framework}: Let $P,Q \in \C[x]$ polynomials s.t. $Q(x) \neq 0\; \forall x \in \R$ and \\
define $R(x):=\frac{P(x)}{Q(x)}$ rational and meromorphic function on $\C$.\\
\vspace{0.2cm}
\begin{itemize}
\item[(1)] If deg$(Q) \geq$ deg$(P)+2$, then
\begin{block}{}
$$\int_{-\infty}^\infty R(x)\; \text{d}x=2\pi i \sum_{Im(a)>0, Q(a)=0} Res_a(R(z))=-2\pi i _{Im(a)<0, Q(a)=0} Res_a(R).$$
\end{block}
In particular, if $R$ is even ($R(x)=R(-x)$) then
$$\int_{0}^\infty R(x)\; \text{d}x=2\pi i \sum_{Im(a)>0} Res_a(R(z)).$$
\end{itemize}
\end{frame}


\begin{frame} \frametitle{Application 2}
\begin{itemize}
\item[(2)] If deg$(Q) \geq$ deg$(P)+1$, then
\begin{block}{}
$$\lim_{r\rightarrow \infty} \int_{-r}^r R(x) e^{ix}\; \text{d}x=2\pi i \sum_{Im(a)>0} Res_a(R(z)e^{iz}).$$
\end{block}
In particular, if $P,Q\in \R[x]$ the
\begin{eqnarray*}
\lim_{r\rightarrow \infty} \int_{0}^r R(x) \cos(x)\; \text{d}x=\pi i \sum_{Im(a)>0} Res_a(R(z)e^{iz}) &\; \text{if } R \text{ even}& \\
\lim_{r\rightarrow \infty} \int_{0}^r R(x) \sin(x)\; \text{d}x=\pi \sum_{Im(a)>0} Res_a(R(z)e^{iz}) &\; \text{if } R \text{ odd}&
\end{eqnarray*}
\end{itemize}
\underline{Example}: $\lim_{r\rightarrow \infty} \int_{0}^r \frac{\sin(x)}{x}\; \text{d}x=\frac{\pi}{2} Res_0(\frac{e^{iz}}{z}) =\frac{\pi}{2}.$
\end{frame}

\begin{frame} \frametitle{Application 3 and 4}
\begin{itemize}
\item[(3)] If $Q(x)\neq 0\; \forall x \in [0,\infty)$ and deg$(Q) \geq$ deg$(P)+2$, then
\begin{block}{}
$$ \int_{0}^\infty R(x)\; \text{d}x=-\sum_{a} Res_a(R(z)\overline{\log}(z)),$$
\end{block}
with $\overline{\log}: \C \backslash [0,\infty) \rightarrow \C, \overline{\log}(re^{i\varphi})=\log(r)+i\varphi$, for $\varphi\in (0, 2\pi)$.\\
\vspace{0.2cm}
\item[(4)] If $Q(x)\neq 0\; \forall x \in [0,\infty)$ s.t. $x=0$ is a zero of most first order and \\deg$(Q) \geq$ deg$(P)+2$, then
\begin{block}{}
$$ \int_{0}^\infty R(x) x^\lambda\; \text{d}x=\frac{2\pi i}{1-e^{2\pi i \lambda}} \sum_{a \neq [0, \infty)} Res_a(R(z) z^\lambda),$$
\end{block}
where $\lambda \in (0,1)$.
\end{itemize}
\end{frame}

\begin{frame} \frametitle{Application 5}
 \underline{Framework}: Let $f: \R \rightarrow \C$ be a $2\pi$- periodic
\begin{itemize}
\item[(5)] Assume that $f$ is given by the restriction of a hol. fct to the unit circle $S^1=\{z \in \C | |z|=1\}.$\\
Then $\exists F:B(0,R)\rightarrow \C$ hol. up to isolated sings. in the disc of $B(0,1)$ with $f(x)=F(e^{ix})$ and 
\begin{block}{}
$$ \int_{0}^{2\pi} f(x)\; \text{d}x=2\pi\sum_{|a|<1} Res_a\Bigg(\frac{F(z)}{z}\Bigg).$$
\end{block}
\item[(5*)] (special case) If $f$ is rational fct in $\cos(x), \sin(x)$:
$$f(x)=R(\cos(x),\sin(x))$$
with
$R(y_1,y_2)=\frac{P(y_1,y_2)}{Q(y_1,y_2)} \neq \infty$ if $y_1,y_2 \in \R$ and $y_1^2+y_2^2=1$, then
\begin{block}{}
$$ \int_{0}^{2\pi} R(\cos(x),\sin(x))\; \text{d}x=2\pi \sum_{|a|<1} Res_{a} \Bigg\{\frac{1}{z}\Bigg[\frac{1}{2}\Big(z+\frac{1}{z}\Big),\frac{1}{2i}\Big(z-\frac{1}{z}\Big)\Bigg]\Bigg\}.$$
\end{block}
\end{itemize}
\end{frame}

\begin{frame} \frametitle{Overview of this chapter}

\end{frame}


}


\end{document}