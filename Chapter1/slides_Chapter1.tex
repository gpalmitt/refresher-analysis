\documentclass[10pt]{beamer}
\usepackage[utf8]{inputenc}
%\usefonttheme{structuresmallcapsserif}
%\usetheme{Madrid}
\reversemarginpar
\usetheme{Madrid}
\usecolortheme{whale}
\usepackage{amsthm}
\usepackage{amssymb}
\usepackage{xcolor}
\usepackage{eurosym}
\usepackage{pifont}
\usepackage{fancybox}
\usepackage{multicol}
\usepackage{amsmath, mathtools}
\usepackage{amssymb}
\usepackage{graphicx}
\usepackage{multirow}
\usepackage{color}
\usepackage{graphicx}
\usepackage{booktabs}
\usepackage{array}
\usepackage{pifont, amsfonts, eufrak}
\usepackage{multicol}
\usepackage{caption}
\usepackage{amssymb,latexsym } 
\usepackage{rotating}
\usepackage{bbm}
\usepackage{cases}
\usepackage{pifont}
%\usepackage{mathabx}
\usepackage{dsfont}
%\usepackage{fourier, heuristica}
\usepackage{array, booktabs}
%\DeclareCaptionFont{blue}{\color{LightSteelBlue3}}

\newcommand{\foo}{\color{blue}\makebox[0pt]{\textbullet}\hskip-0.5pt\vrule width 1pt\hspace{\labelsep}}

\usepackage{etoolbox}

\newcommand\Pitem{%
  \addtocounter{enumi}{-1}%
  \renewcommand\theenumi{\arabic{enumi}'}%
  \item%
  \renewcommand\theenumi{\arabic{enumi}}%
}

\definecolor{or}{rgb}{1.0, 0.4, 0.0}
\definecolor{greng}{rgb}{0.0, 0.62, 0.38}
\definecolor{gro}{rgb}{0.77, 0.76, 0.75}
\definecolor{blo}{rgb}{0.15, 0.38, 0.61}
\definecolor{roud}{rgb}{0.81, 0.06, 0.13}

%\usepackage{lmodern}
%\newcommand\itemshape[1]{%
%\setbeamertemplate{itemize item}[#1]{%
%\usebeamertemplate{itemize item}
%}

\def\X#1{%
        %#1%
        %\textcircled{#1}%
        \raisebox{.9pt}{\textcircled{\raisebox{-.9pt}{#1}}}%
        %\ding{\numexpr171+#1\relax}%
}


\usepackage{tikz}
\usepackage{tikz-cd}
\usetikzlibrary{arrows}
\usetikzlibrary{positioning}
\usepackage{caption}
\usetikzlibrary{shapes,arrows}
\usepackage[all]{xy}

\setbeamertemplate{navigation symbols}{}

\DeclareMathOperator{\spn}{span}
\newcommand{\R}{\mathbb{R}}
\newcommand{\C}{\mathbb{C}}
\newcommand{\N}{\mathbb{N}}
\newcommand{\Z}{\mathbb{Z}}
\newcommand{\E}{\mathbb{E}}
\newcommand{\F}{\mathbb{F}}
\newcommand{\HH}{\mathbb{H}}
\newcommand{\VV}{\mathbb{V}}
\newcommand{\cc}{\mathbf{c}}
\newcommand{\CC}{\mathbf{C}}

\newcommand{\U}{\mathcal{U}}
\newcommand{\DD}{\mathcal{D}}
\newcommand{\g}{\mathfrak{g}}
\newcommand{\kk}{\mathfrak{k}}
\newcommand{\aL}{\mathfrak{a}}
\newcommand{\End}{\text{End}}
\newcommand{\Hom}{\text{Hom}}
\newcommand{\Hol}{\text{Hol}}
\newcommand{\Pol}{\text{Pol}}
\newcommand{\opp}{\text{opp}}
\newcommand{\Ad}{\text{Ad}}
\newcommand{\Ker}{\text{Ker}}
\newcommand{\Impart}{\text{Im}}
\newcommand{\Repart}{\text{Re}}
\newcommand{\supp}{\text{supp}}
\newcommand{\Mat}{\text{Mat}}
\newcommand{\Tr}{\text{Tr}}
\newcommand{\Id}{\text{Id}}
\newcommand{\rk}{\text{rk}}
\newcommand{\diag}{\text{diag}}
\newcommand*{\QEDA}{\null\nobreak\hfill\ensuremath{\square}}%


\setbeamertemplate{theorems}[numbered] 

\AtBeginSection[]
{
  \begin{frame}
    \tableofcontents[currentsection]
  \end{frame}
}
\logo{\includegraphics[height=1cm]{dmath}}
\title[Refresher Courses in Analysis] %optional
{Chapter 1: Basic of real analysis of several variables}

\subtitle{Refresher Courses in Analysis}
\author[Guendalina \textsc{Palmirotta}]{Guendalina \textsc{Palmirotta}}
\institute[]{University of Luxembourg, Department of Mathematics}
\date[]{\today}
\newcommand{\nologo}{\setbeamertemplate{logo}{}}
\begin{document}


\begin{frame}
\titlepage
\end{frame}

{\nologo


%\begin{frame}
%\frametitle{Contents}
%\tableofcontents
%\end{frame}

\section{1.1.  Banach's fixed point theorem}

\begin{frame} \frametitle{Motivation example}
We already know that the equation 
$$x^2=2$$
has a unique solution $x\in (0,\infty)$, called 
$$\sqrt{2}=1.41421356237...$$
In fact, we know that \\
\vspace{0.2cm}
There is an alternative way to \textbf{construct} $\sqrt{2}$:\\
\vspace{0.2cm}
\underline{Step 1}: Reformulate the eq. $x^2=2$ as ``\textit{fixed point problem}"
\begin{block}{} 
\vspace{1cm}
\end{block}{}

\end{frame}

\begin{frame} \frametitle{Motivation example}
\underline{Step 2}: Choose $x_0 \in [1,2]$ an define recursively
\begin{block}{} 
\vspace{0.5cm}
\end{block}{}
\vspace{0.2cm}
\underline{Step 3}:
Estimate. let $x,y \in [1,2]$,  then 
\begin{block}{} 
\vspace{0.5cm}
\end{block}{}
\end{frame}

\begin{frame} \frametitle{Contracting function}
\begin{alertblock}{Definiton 1 (contracting function)}
Let $(X,d)$ be a metric space and let $f:X \rightarrow X$ be a map.\\
We say that $f$ is \textbf{contracting} if there exists a constant $C\in [0,1)$ such that $\forall x,y \in X$ we have
$$d(f(x),f(y)) \leq C\cdot d(x,y).$$
\end{alertblock}
\underline{Remarks}
\begin{itemize}
\item Def.~1 can be generalized as follows:\\
\vspace{2cm}
\item $f$ is contracting $\Rightarrow f$ is continuous (even Lipschitz continuous).
\item $f$ is Lipschitz continuous $\Rightarrow f$ is continuous.
\end{itemize}
\end{frame}

\begin{frame}\frametitle{Banach's fixed point theorem}
\begin{block}{Theorem 1 (Banach's fixed point theorem)}
Let $(X,d)$ be a complete metric space and let $f:X \rightarrow X$ be a contracting map.\\
\textbf{Then} $f$ has a unique fixed point $\overline{x}\in X$.\\
In other words, the equation $f(x)=x$ has exactly one solution $\overline{x}\in X$.\\
\vspace{0.2cm}
\textbf{Moreover}, for any $x_0 \in X$, the sequence $(f^n(x_0))_n$, where $f^n(x)=f \circ \dots \circ f(x)$, converge to the fixed point $\overline{x}\in X$.
\end{block}
\underline{Sketch of the proof}\\
\vspace{4cm}
\end{frame}

\begin{frame}\frametitle{Banach's fixed point theorem}
\end{frame}


\begin{frame}\frametitle{Banach's fixed point theorem}
\underline{Remarks}
\begin{itemize}
\item Thm.~1 shows that our ``construction'' of $\sqrt{2}$ was correct.
\item Thm.~1 also provides a practical method to solve equations like $f(x)=x$\\(at least approx. by computers).
\item Thm.~1 does NOT work for ``Lipschitz constant" $L=1$ in general.\\
\vspace{0.2cm}
\textbf{Simplest example}: 
\end{itemize}
\end{frame}




\section{1.2.  Implicit function theorem}


\begin{frame} \frametitle{Preparation}
Let 
\begin{itemize}
\item[] $U \subset \R^p \times \R^q$, $(p,q \geq 1)$, be open and
\item[] $F: U \rightarrow \R^q$ be a cont. diff. map.
\end{itemize}
We consider the equation
\begin{block}{} 
\begin{equation} \label{eq:1}
\;
\end{equation}
\end{block}{}
Given $x$, we want to find $y$ s.t.  (\ref{eq:1}) is satisfied.\\
For fixed $x$, (\ref{eq:1}) is a system of $p$ equations in $q$ variables.\\
\vspace{0.2cm}
\textbf{General question}:
\begin{center} Under which conditions can we solve (\ref{eq:1}) uniquely if possible? \end{center}
\vspace{0.2cm}
\underline{Find} a solution $y=:g(x)$ depending on $x$ s.t. $F(x,g(x))=0$.\\
Ideally, we want that $g$ is a ``nice'' function, that means cont. diff.\\
We may ask: 
\begin{center} What can be said about it differential $dg(x)$? \end{center}
\end{frame}


\begin{frame} \frametitle{Preparation}
\underline{Recall the solution}: $U \subset \R^p \times \R^q,  F: U \rightarrow \R^q$ of class $\mathcal{C}^1$.\\
We want to \textbf{find} a solution function (``\textcolor{blue}{implicit function}'') defined on some open subset $U_1 \subset \R^p$ with values in another open subset $U_2 \subset \R^q$
\begin{block}{} 
\vspace{0.5cm}
\begin{equation*}
\;
\end{equation*}
\end{block}{}
\vspace{0.2cm}
\underline{A necessary condition for the existence of a $\mathcal{C}^1$-solution}\\
Let $(x,y) \in U$ be fixed.\\
We consider the differential of $F$ at $(x,y)$
$$dF(x,y):\R^p \times \R^q \rightarrow \R \text{ linear map}.$$
$dF(x,y)$ can be written as a sum of 2 maps. Let
\begin{itemize}
\item[] $p_1:\R^p \times \R^q \rightarrow \R^p$ and 
\item[] $p_1:\R^p \times \R^q \rightarrow \R^q$ be the projections.
\end{itemize}
\end{frame}


\begin{frame} \frametitle{A necessary condition for the existence of a $\mathcal{C}^1$-solution}
Then
\begin{block}{} 
\vspace{1.5cm}
\end{block}{}
In matrix notation, we have\\
\vspace{3cm}

\end{frame}

\begin{frame} \frametitle{A necessary condition for the existence of a $\mathcal{C}^1$-solution}
Let  $U_1 \times U_2 \subset U,$ and $g:U_1 \rightarrow U_2$ be differentiable.\\
We consider the function $f:U_1 \rightarrow \R^q$ given by
$$\;$$
In other words: 
$\Rightarrow f$ is differentiable,  and by the chain rule we obtain
\begin{block}{} 
\vspace{0.8cm}
\end{block}{}
If $g$ is a solution of (1a), then $f(x)=0 \;\forall x\in U_1$, therefore $df(x)=0 \;\forall x \in U_2$.
$\Rightarrow dg(x)$ is a solution of
\begin{block}{} 
\vspace{0.5cm}
\end{block}{}
\begin{itemize}
\item If the linear map $d_2F(x,g(x))$ is invertible, then we get a unique solution:
\begin{block}{} 
\vspace{0.5cm}
\end{block}{}
\item If $d_2F(x,g(x))$ is NOT invertible, then it could happen that (2a) has NO solution.
\end{itemize}
\end{frame}


\begin{frame} \frametitle{Implicit function theorem}
\begin{block}{Theorem 2 (Implicit function theorem)}
Assume: $U \subset \R^p \times \R^q$ open and $F:U \rightarrow \R^q$ cont. diff.  (i.e. $F \in \mathcal{C}^1(U,\R^q)$).\\
If $(x_0,y_0) \in U$ s.t.
\begin{itemize}
\item[(a)] $F(x_0,y_0)=0$,
\item[(b)] $d_2F(x_0,y_0)$ is invertible, i.e.  det$(d_2F(x_0,y_0)) \neq 0$,
\end{itemize}
\textbf{then} $\exists$ open neighb. $U_1 \subset \R^p$ of $x_0$ and $U_2 \subset \R^q$ of $y_0$ with $U_1 \times U_2 \subset U$ s.t.
\begin{itemize}
\item[(i)] for each $x\in U_1$ there is a unique $y=g(x) \in U_2$ with $F(x,g(x))=0$,
\item[(ii)] the function $g:U_1 \rightarrow U_2$ defined by (1) is cont. diff.
\end{itemize}
Hence $d_2F(x,g(x))$ is invertible $\forall x\in U_1$ and
$$dg(x)=-[d_2F(x,g(x))]^{-1} \cdot d_1 F(x,g(x)).$$
\end{block}
\underline{Sketch of the proof:}
Resets of Banach's fixed point theorem (Thm.~1). $\qed$
\end{frame}




\section{1.3.  Inverse function theorem}
\begin{frame}
\frametitle{Problem}
\underline{Framework}: $U \subset \R^q$ be open and $f:U \rightarrow  \R^q$ be a $\mathcal{C}^1$-function.\\
\vspace{0.2cm}
We \textbf{look} for an inverse function
\begin{block}{} 
\vspace{2cm}
\end{block}{}
We can rewrite (i) as an ``\textcolor{blue}{implicit function problem}", namely
\begin{block}{} 
\vspace{0.5cm}
\end{block}{}
\end{frame}

\begin{frame}
\frametitle{Inverse function theorem}
\begin{block}{Theorem 3 (Inverse function theorem)}
Assume: $U \subset \R^n$ open, $f: U \rightarrow \R^n$ cont. diff., $x_0 \in U$ s.t. 
$df(x_0)$ \text{ is invertible}.
\textbf{Then}: $\exists$ open neighbs.  $U_0 \subset U$ of $x_0$ and $V=f(U) \subset \R^n$ of $f(x_0)$,
and a cont. diff. function $$g=(f|_{U_0})^{-1}:V \rightarrow U_0.$$
\vspace{0.2cm}
\textbf{Moreover},
$$dg(f(x))=[df(x)]^{-1} \; \forall x \in U_0.$$
\end{block}
\underline{Remark}:
\vspace{1cm}
\end{frame}

\begin{frame}
\frametitle{Application of Thm.~3: Diffeomorphismus}
\begin{alertblock}{Definition 2 (diffeomorphism)}
Let $U,V \subset \R^n$ be open and $f:U \rightarrow V$ be a map of class $\mathcal{C}^k$, $k\in \N^*$ (or of class $\mathcal{C}^\infty$).\\
$f$ is called \textbf{diffeomorphism} of class $\mathcal{C}^k$ (or $\mathcal{C}^\infty$) $\iff$ 
\begin{itemize}
\item[(i)] $f$ is bijective and 
\item[(ii)] $f^{-1}: V \rightarrow U$ is also of class $\mathcal{C}^k$ (or $\mathcal{C}^\infty$).
\end{itemize}
\end{alertblock}
\vspace{0.2cm}
\begin{block}{Proposition 1}
Let $U,V \subset \R^n$ be open and $f:U \rightarrow V$ bijective of class $\mathcal{C}^k$ (or $\mathcal{C}^\infty$).\\
\textbf{Then},  the following conditions are equivalent:
\begin{itemize}
\item[(a)] $df(x)$ is an invertible map $\forall x\in U.$
\item[(b)] $f^{-1}: V \rightarrow U$ is differentiable.
\item[(c)] $f: U \rightarrow V$ is a diffeomorphism of $\mathcal{C}^k$.
\end{itemize}
\end{block}
\end{frame}


\begin{frame}
\frametitle{Examples}
\begin{itemize}
\item[\textcolor{blue}{(1)}] (example of a bijective $\mathcal{C}^\infty$ map that is \textbf{not} a diffeomorphism)\\
\vspace{3cm}
\item[\textcolor{blue}{(2)}] (polar coord. in the plane $\R^2$)\\
\vspace{3cm}
\end{itemize}
\end{frame}


\begin{frame}
\frametitle{Overview of this chapter}
\end{frame}

}


\end{document}