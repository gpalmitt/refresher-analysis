\documentclass[10pt]{beamer}
\usepackage[utf8]{inputenc}
%\usefonttheme{structuresmallcapsserif}
%\usetheme{Madrid}
\reversemarginpar
\usetheme{Madrid}
\usecolortheme{whale}
\usepackage{amsthm}
\usepackage{amssymb}
\usepackage{xcolor}
\usepackage{eurosym}
\usepackage{pifont}
\usepackage{fancybox}
\usepackage{multicol}
\usepackage{amsmath, mathtools}
\usepackage{amssymb}
\usepackage{graphicx}
\usepackage{multirow}
\usepackage{color}
\usepackage{graphicx}
\usepackage{booktabs}
\usepackage{array}
\usepackage{pifont, amsfonts, eufrak}
\usepackage{multicol}
\usepackage{caption}
\usepackage{amssymb,latexsym } 
\usepackage{rotating}
\usepackage{bbm}
\usepackage{cases}
\usepackage{pifont}
%\usepackage{mathabx}
\usepackage{dsfont}
%\usepackage{fourier, heuristica}
\usepackage{array, booktabs}
%\DeclareCaptionFont{blue}{\color{LightSteelBlue3}}

\newcommand{\foo}{\color{blue}\makebox[0pt]{\textbullet}\hskip-0.5pt\vrule width 1pt\hspace{\labelsep}}

\usepackage{etoolbox}

\newcommand\Pitem{%
  \addtocounter{enumi}{-1}%
  \renewcommand\theenumi{\arabic{enumi}'}%
  \item%
  \renewcommand\theenumi{\arabic{enumi}}%
}

\definecolor{or}{rgb}{1.0, 0.4, 0.0}
\definecolor{greng}{rgb}{0.0, 0.62, 0.38}
\definecolor{gro}{rgb}{0.77, 0.76, 0.75}
\definecolor{blo}{rgb}{0.15, 0.38, 0.61}
\definecolor{roud}{rgb}{0.81, 0.06, 0.13}

%\usepackage{lmodern}
%\newcommand\itemshape[1]{%
%\setbeamertemplate{itemize item}[#1]{%
%\usebeamertemplate{itemize item}
%}

\def\X#1{%
        %#1%
        %\textcircled{#1}%
        \raisebox{.9pt}{\textcircled{\raisebox{-.9pt}{#1}}}%
        %\ding{\numexpr171+#1\relax}%
}


\usepackage{tikz}
\usepackage{tikz-cd}
\usetikzlibrary{arrows}
\usetikzlibrary{positioning}
\usepackage{caption}
\usetikzlibrary{shapes,arrows}
\usepackage[all]{xy}

\setbeamertemplate{navigation symbols}{}

\DeclareMathOperator{\spn}{span}
\newcommand{\R}{\mathbb{R}}
\newcommand{\C}{\mathbb{C}}
\newcommand{\N}{\mathbb{N}}
\newcommand{\Z}{\mathbb{Z}}
\newcommand{\E}{\mathbb{E}}
\newcommand{\F}{\mathbb{F}}
\newcommand{\HH}{\mathbb{H}}
\newcommand{\VV}{\mathbb{V}}
\newcommand{\cc}{\mathbf{c}}
\newcommand{\CC}{\mathbf{C}}

\newcommand{\U}{\mathcal{U}}
\newcommand{\DD}{\mathcal{D}}
\newcommand{\g}{\mathfrak{g}}
\newcommand{\kk}{\mathfrak{k}}
\newcommand{\aL}{\mathfrak{a}}
\newcommand{\End}{\text{End}}
\newcommand{\Hom}{\text{Hom}}
\newcommand{\Hol}{\text{Hol}}
\newcommand{\Pol}{\text{Pol}}
\newcommand{\opp}{\text{opp}}
\newcommand{\Ad}{\text{Ad}}
\newcommand{\Ker}{\text{Ker}}
\newcommand{\Impart}{\text{Im}}
\newcommand{\Repart}{\text{Re}}
\newcommand{\supp}{\text{supp}}
\newcommand{\Mat}{\text{Mat}}
\newcommand{\Tr}{\text{Tr}}
\newcommand{\Id}{\text{Id}}
\newcommand{\rk}{\text{rk}}
\newcommand{\diag}{\text{diag}}
\newcommand*{\QEDA}{\null\nobreak\hfill\ensuremath{\square}}%


\setbeamertemplate{theorems}[numbered] 

\AtBeginSection[]
{
  \begin{frame}
    \tableofcontents[currentsection]
  \end{frame}
}
\logo{\includegraphics[height=1cm]{dmath}}
\title[Refresher Courses in Analysis] %optional
{Chapter 0: Some brief recalls (``Sandbox")}

\subtitle{Refresher Courses in Analysis}
\author[Guendalina \textsc{Palmirotta}]{Guendalina \textsc{Palmirotta}}
\institute[]{University of Luxembourg, Department of Mathematics}
\date[]{\today}
\newcommand{\nologo}{\setbeamertemplate{logo}{}}
\begin{document}


\begin{frame}
\titlepage
\end{frame}

{\nologo


%\begin{frame}
%\frametitle{Contents}
%\tableofcontents
%\end{frame}

\section{0.1.  Some recall on metric spaces}

\begin{frame}
\frametitle{Definitions and properties of metric spaces}
\begin{alertblock}{Definition 1}
\begin{itemize}
\item (distance) Let $X$ be a non-empty space.
A \textbf{distance} $d: X \times X \rightarrow [0, \infty)$ verifies the 3 properties: $\forall x,y,z \in X$:
\begin{itemize}
\item[(i)] symmetry: $d(x,y)=d(x,y)$.
\item[(ii)] pos.  definiteness: $d(x,y)=0 \iff x=y$.
\item[(iii)] $\Delta$-inequality: $d(x,y) \leq d(x,z)+d(z,y).$
\end{itemize}
\item (metric space) If $d$ is a distance on $X$, then $(X,d)$ is a \textbf{metric space}.
\item (Cauchy seq.) $(x_n)_{n\in \N}$ is a \textbf{Cauchy seq. }:
$\forall \epsilon >0 \;\exists N \in \N$ s.t.  $\forall n,m \in \N$
$$d(x_n,x_m) < \epsilon.$$
\item (completeness) A metric space $(X,d)$ is \textbf{complete} if every Cauchy seq. in $X$ converges in $X$.
\end{itemize}
\end{alertblock}
\vspace{0.2cm}
\begin{block}{Proposition 1}
Let $(X,d)$ be a complete metric space.\\
If $Y$ is a closed subset of $X$ then $(Y,d)$ is a complete metric space.
\end{block}
\end{frame}



\begin{frame}
\frametitle{Continuity and Lipschitz continuity}
\begin{alertblock}{Definition 2}
Let $(X,d_x)$ and $(Y,d_y)$ be 2 metric spaces. 
\begin{itemize}
\item (continuity) A function $f:X \rightarrow Y$ is \textbf{continuous} in $x_0 \in X$ if $\forall \epsilon >0 \; \exists \delta >0$ s.t.  $d_x(x_0,x) < \delta$ then $$d_y(f(y_0),f(x)) < \epsilon.$$
If $f$ is continuous in $x_0, \; \forall x_0\in X$, we say that $f$ is continuous on $X$.
\item (Lipschitz cont.) A function  $f:X \rightarrow Y$ is \textbf{Lipschitz continuous} $\iff \exists$ a constant $L \in [0,\infty)$ s.t. $\forall x_1,x_2 \in X$, we have
$$d_y(f(x_1),f(x_2)) \leq L d_x(x_1,x_2).$$
L is called the \textbf{Lipschitz constant}.
\end{itemize}
\end{alertblock}
\end{frame}



%\begin{frame}
%\frametitle{Metric spaces of bounded continuous functions}
%\textcolor{red}{[to do]}
%\end{frame}


\section{0.2.  Some recall on normed vector spaces and operators}

\begin{frame}
\frametitle{Definitions and properties of normed vector spaces}
\begin{alertblock}{Definition 3}
Let $V$ be a non-empty vector space (v.s.).
\begin{itemize}
\item (norm) A \textbf{norm} on $V$ is a function $||\cdot||:V \rightarrow [0, \infty)$ satisfying the 3 axioms: $\forall x,y \in V$ and $\lambda \in \R$
\begin{itemize}
\item[(i)] pos. definiteness: $||x||=0 \Rightarrow x=0$.
\item[(ii)] absolute homogeneity: $||\lambda x||=|\lambda| ||x||$.
\item[(iii)] $\Delta$-ineq. : $||x+y||\leq ||x||+||y||$.
\end{itemize}
\item (normed v.s.) If $||\cdot||$ is a norm on $V$, then $(V,||\cdot||)$ is a \textbf{normed v.s.}.
\item (Banach space) A \textbf{Banach space} is complete normed v.s.
\item (linear maps and operators) Let $(V,||\cdot||_V), (W,||\cdot||_W)$ be 2 finite-dimensional normed v.s. over $\R$ or $\C$.  Denote by $\mathcal{L}(V,W)$ be the \textbf{v.s. of linear map} $A:V \rightarrow W$.
We define the \textbf{operator} of $A$ by
$$||A||_{\text{op}}=||A||:=\sup\{||Av||_{W}| v\in V, ||v||=1\}.$$
\end{itemize}
\end{alertblock}
\end{frame}

\begin{frame}
\frametitle{Examples}
\begin{itemize}
\item[\textcolor{blue}{(1)}] Typical distances:
\begin{eqnarray*}
d_2:(x,y) &\mapsto& \sqrt{\sum_{i=1}^n |x_i-y_i|^2}, \;n\in \N\text{ (Euclidean distance on $\R^n$)}\\
d_p:(x,y) &\mapsto& \Big(\sum_{i=1}^n |x_i-y_i|^p\Big)^{1/p}, \;p\in [1,\infty) \text{ (generalization)}\\
d_\infty: (x,y) &\mapsto& \max_{i\in\{1, \dots,n\}} |x_i-y_i|, \;n\in \N.
\end{eqnarray*}
\item[\textcolor{blue}{(2)}] The corresponding normed distances:
\begin{eqnarray*}
||\cdot||_2=d_2:x &\mapsto& \sqrt{\sum_{i=1}^n |x_i|^2}, \;n\in \N\text{ (Euclidean norm on $\R^n$)}\\
||\cdot||_p=d_p:x &\mapsto& \Big(\sum_{i=1}^n |x_i|^p\Big)^{1/p}, \;p\in [1,\infty)\\
||\cdot||_\infty=d_\infty: x &\mapsto& \max_{i\in\{1, \dots,n\}} |x_i|,  \;n\in \N.
\end{eqnarray*}
\end{itemize}
\end{frame}

\begin{frame}
\frametitle{Some important properties}
\begin{itemize}
\item Property of normed operators
\end{itemize}
\begin{block}{Proposition 2}
Let $(V,||\cdot||_V), (W,||\cdot||_W)$ 2 f.-d.  normed v.s. over $\R$ or $\C$ and 
$||A||_{\mathcal{L}(V,W)} <\infty \; \forall A \in \mathcal{L}(V,W)$.
\textbf{Then},  for every $v\in V$, we have
$$||Av||_W \leq ||A||_{\mathcal{L}(V,W)} \;||v||_W.$$
\end{block}
\begin{itemize}
\item Connection between diff. and Lipschitz continuity
\end{itemize}
\begin{block}{Lemma 1}
Let $U\subset \R^n$ be open, $f:U \rightarrow \R^k$ diff. fct. and $A \subset U$ convex s.t.
$df|_A: A \rightarrow \mathcal{L}(\R^n,\R^k)$ \text{ is bounded},  i.e. $\exists C \; ||df(x)||_{\mathcal{L}(\R^n,\R^k)} \leq C \; \forall x\in A$.\\
\textbf{Then}, $f$ is Lipschitz cont. on $A$ with Lipschitz constant
$$L=\sup_{x\in A} ||df(x)||_{\mathcal{L}(\R^n,\R^k)}.$$
\underline{In particular},  if $f$ is cont. diff., then $f$ is Lipschitz cont on each \textbf{compact} subset $A \subset U$, e.g. on closed balls.
\end{block}
\end{frame}






}


\end{document}