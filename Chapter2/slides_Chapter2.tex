\documentclass[10pt]{beamer}
\usepackage[utf8]{inputenc}
%\usefonttheme{structuresmallcapsserif}
%\usetheme{Madrid}
\reversemarginpar
\usetheme{Madrid}
\usecolortheme{whale}
\usepackage{amsthm}
\usepackage{amssymb}
\usepackage{xcolor}
\usepackage{eurosym}
\usepackage{pifont}
\usepackage{fancybox}
\usepackage{multicol}
\usepackage{amsmath, mathtools}
\usepackage{amssymb}
\usepackage{graphicx}
\usepackage{multirow}
\usepackage{color}
\usepackage{graphicx}
\usepackage{booktabs}
\usepackage{array}
\usepackage{pifont, amsfonts, eufrak}
\usepackage{multicol}
\usepackage{caption}
\usepackage{amssymb,latexsym } 
\usepackage{rotating}
\usepackage{bbm}
\usepackage{cases}
\usepackage{pifont}
%\usepackage{mathabx}
\usepackage{dsfont}
%\usepackage{fourier, heuristica}
\usepackage{array, booktabs}
%\DeclareCaptionFont{blue}{\color{LightSteelBlue3}}

\newcommand{\foo}{\color{blue}\makebox[0pt]{\textbullet}\hskip-0.5pt\vrule width 1pt\hspace{\labelsep}}

\usepackage{etoolbox}

\newcommand\Pitem{%
  \addtocounter{enumi}{-1}%
  \renewcommand\theenumi{\arabic{enumi}'}%
  \item%
  \renewcommand\theenumi{\arabic{enumi}}%
}

\definecolor{or}{rgb}{1.0, 0.4, 0.0}
\definecolor{greng}{rgb}{0.0, 0.62, 0.38}
\definecolor{gro}{rgb}{0.77, 0.76, 0.75}
\definecolor{blo}{rgb}{0.15, 0.38, 0.61}
\definecolor{roud}{rgb}{0.81, 0.06, 0.13}

%\usepackage{lmodern}
%\newcommand\itemshape[1]{%
%\setbeamertemplate{itemize item}[#1]{%
%\usebeamertemplate{itemize item}
%}

\def\X#1{%
        %#1%
        %\textcircled{#1}%
        \raisebox{.9pt}{\textcircled{\raisebox{-.9pt}{#1}}}%
        %\ding{\numexpr171+#1\relax}%
}


\usepackage{tikz}
\usepackage{tikz-cd}
\usetikzlibrary{arrows}
\usetikzlibrary{positioning}
\usepackage{caption}
\usetikzlibrary{shapes,arrows}
\usepackage[all]{xy}

\setbeamertemplate{navigation symbols}{}

\DeclareMathOperator{\spn}{span}
\newcommand{\R}{\mathbb{R}}
\newcommand{\C}{\mathbb{C}}
\newcommand{\N}{\mathbb{N}}
\newcommand{\Z}{\mathbb{Z}}
\newcommand{\E}{\mathbb{E}}
\newcommand{\F}{\mathbb{F}}
\newcommand{\HH}{\mathbb{H}}
\newcommand{\VV}{\mathbb{V}}
\newcommand{\cc}{\mathbf{c}}
\newcommand{\CC}{\mathbf{C}}

\newcommand{\U}{\mathcal{U}}
\newcommand{\DD}{\mathcal{D}}
\newcommand{\g}{\mathfrak{g}}
\newcommand{\kk}{\mathfrak{k}}
\newcommand{\aL}{\mathfrak{a}}
\newcommand{\End}{\text{End}}
\newcommand{\Hom}{\text{Hom}}
\newcommand{\Hol}{\text{Hol}}
\newcommand{\Pol}{\text{Pol}}
\newcommand{\opp}{\text{opp}}
\newcommand{\Ad}{\text{Ad}}
\newcommand{\Ker}{\text{Ker}}
\newcommand{\Impart}{\text{Im}}
\newcommand{\Repart}{\text{Re}}
\newcommand{\supp}{\text{supp}}
\newcommand{\Mat}{\text{Mat}}
\newcommand{\Tr}{\text{Tr}}
\newcommand{\Id}{\text{Id}}
\newcommand{\rk}{\text{rk}}
\newcommand{\diag}{\text{diag}}
\newcommand*{\QEDA}{\null\nobreak\hfill\ensuremath{\square}}%


\setbeamertemplate{theorems}[numbered] 

\AtBeginSection[]
{
  \begin{frame}
    \tableofcontents[currentsection]
  \end{frame}
}
\logo{\includegraphics[height=1cm]{dmath}}
\title[Refresher Courses in Analysis] %optional
{Chapter 2: Ordinary Differential Equations}

\subtitle{Refresher Courses in Analysis}
\author[Guendalina \textsc{Palmirotta}]{Guendalina \textsc{Palmirotta}}
\institute[]{University of Luxembourg, Department of Mathematics}
\date[]{\today}
\newcommand{\nologo}{\setbeamertemplate{logo}{}}
\begin{document}


\begin{frame}
\titlepage
\end{frame}

{\nologo


%\begin{frame}
%\frametitle{Contents}
%\tableofcontents
%\end{frame}

\section{2.1. Differential equations of first and higher order}

\begin{frame} \frametitle{Explicit and implicit form}
\begin{alertblock}{Definition 1 (Ordinary differential equation)}
Let $\Omega \subset \R \times \R^n \times \cdots \times \R^n$ be open and $F:\Omega \rightarrow \R^n$ be a continuous function.\\
An \textbf{ordinary differential equation} (ODE) of order $k\in \N$ is an equation of the following kind:
\begin{itemize}
\item \;\\
\vspace{0.2cm}
\item
\end{itemize}
\end{alertblock}

\begin{alertblock}{Definition 2}
Let $I \subset \R$ be an open interval and $\Omega, F$ as in Def.~1.\\
A \textbf{solution} of (1) is a function $\varphi: I \rightarrow \R^n$ s.t. $\forall x \in I$:
\begin{itemize}
\item[(i)]\;\\
\vspace{0.2cm}
\item[(ii)]\;\\
\vspace{0.2cm}
\item[(iii)]
\end{itemize}
\end{alertblock}
\end{frame}

\begin{frame} \frametitle{Examples}
\underline{Remarks}:\\
\begin {itemize}
\item Often, we take $n=1$. If $n\geq 2$, one calls (1) also a system of ODEs.
\item The Implicit function theorem tells us that under certain conditions (2) is equivalent to
$$\;$$
at least locally!
\end{itemize}

\vspace{0.5cm}
\underline{Examples}:\\
\vspace{0.2cm}
\begin {itemize}
\item Explicit form:\\
\vspace{0.2cm}
\item Implicit form:
\end{itemize}
\end{frame}

\begin{frame} \frametitle{Cauchy problem}
\underline{Aim}: We are interested in \textbf{finding solutions} of
\begin{block}{}
\vspace{0.5cm}
\end{block}
\textbf{satisfying initial conditions}.

\begin{alertblock}{Defintion 3 (Initial value problem or Cauchy problem (CP))}
Given $x_0\in \R$ and $a_0,a_1, \dots, a_{k-1} \in \R^n$. We have to find an open interval $I_0$ containing $x_0 \in \R$ and a function $\varphi \in \mathcal{C}^k(I_0,\R^n)$ s.t.
\begin{itemize}
\item[(i)] $\varphi$ is a solution of (1).
\item[(ii)] $\varphi^{(l)}(x_0)=a_l,\; l=0, \dots,k-1$.
\end{itemize}
We say that $(x_0,a_0,a_1, \dots, a_{k-1})$ is the \textbf{initial condition} of the Cauchy problem or initial value problem.
\end{alertblock}
\end{frame}


\begin{frame} \frametitle{Some examples}
\begin{itemize}
\item[\textcolor{blue}{1.}]\; \\
\vspace{7cm}
\end{itemize}
\end{frame}

\begin{frame} \frametitle{Some examples}
\begin{itemize}
\item[\textcolor{blue}{2.}] (Newton's law in classical mechanics, $k=2,n=3$)\\
\vspace{7cm}
\end{itemize}
\end{frame}



\section{2.2 Reduction of order}

\begin{frame} \frametitle{Problem}
\underline{Framework}: $U \subset \R^n$ open subset, $I \subset \R$ interval s.t. $I \times U \subset \Omega$ and
\begin{block}{}
\vspace{0.4cm}
\end{block}
We define a \textbf{new} function 
\begin{block}{}
\vspace{0.4cm}
\end{block}
Then $Y$ satisfies the following system of diff. eqs.:\\
\vspace{3cm}
\underline{Fact}: If $y$ solves the CP associate with (4) with $y^{(l)}(x_0)=a_l, l=0, \cdots, n-1$, \\
then $Y$ solves the CP with $Y(x_0)=(a_0, a_1, \cdots, a_{k-1})$.\\
\underline{Vice versa}: If $Y$ solves (5) then $y:=Y_0$ solves (4).
\end{frame}

\begin{frame} \frametitle{Reduction of order}
\begin{block}{Proposition 1}
Let $k\in \N$ and $\Omega \subset \R \times \R^n \times \cdots \times \R^n$ be an open interval. Consider a function $F: \Omega \rightarrow \R^n$.\\
There is a \textbf{one-to-one correspondence} between the solutions (1) and solutions of the equation\\
\vspace{2cm}
\end{block}
\vspace{0.2cm}
\underline{Moral}: For the understanding of differential equations (DEs) of arbitrary order it is (in principle) sufficient to understand (system of) DEs of first order. One has to pay the price that one has to introduce more variables!\\
\end{frame}

\begin{frame} \frametitle{Reduction of order}
\underline{Remarks}:
\begin{itemize}
\item Most of the diff. eqs. are not explicitly solvable. Nevertheless, we want to understand
\begin{itemize}
\item \;\\
\vspace{0.2cm}
\item\;\\
\vspace{0.2cm}
\item\;\\
\vspace{0.2cm}
\end{itemize}
\item Prop.~1 tells us that it is sufficient to understand 1. order eqs.
\item In Example 2.,  by applying the reduction  of order to (3) (which is of order 2 on $\R^3$), we obtain an eq. of 1. order in
$$\;$$
\end{itemize}
\end{frame}



\section{2.3. Elementary solutions for certain equations of first order in one variable}

\begin{frame}\frametitle{Elementary solutions of the Cauchy problem}
\underline{Aim}: We are interested in \textbf{finding elementary solutions} of differential equations of first order
\begin{block}{}
\vspace{0.5cm}
\end{block}
satisfying initial conditions.
\vspace{0.4cm}

We will focus on the following:
\vspace{0.2cm}
\begin{itemize}
\item[\ding{228}] Separation of variables
\item[\ding{228}] Linear equations and variation of constant
\item[\ding{228}] ``Homogeneous" differential equation
\end{itemize}
\end{frame}

\begin{frame}\frametitle{Separation of variables}
Consider a diff. eq.
\begin{block}{}
\vspace{0.8cm}
\end{block}
\vspace{0.5cm}

\begin{itemize}
\item[a)] If $g(y_0) \neq 0$, then\\
\vspace{2cm}
\item[b)] If $g(y_0) = 0$, then\\
\vspace{2cm}
\end{itemize}
\end{frame}

\begin{frame}\frametitle{Example}
Consider a simple \underline{non-linear} eq. of first order ($k=1$) with $n=1$:
\begin{block}{}
\vspace{1cm}
\end{block}
Such an eq. in one variable can be solved by the method of separation of variables.
\begin{itemize}
\item \textbf{Case $a=0$}:\\
\vspace{0.2cm}
\item \textbf{Case $a\neq 0$}:\\
\vspace{3cm}
\end{itemize}
For non-linear eqs. solutions need not to be defined everywhere. The (maximal) domain of definition may depend on the initial condition.
\end{frame}

\begin{frame}\frametitle{Linear equations and variation of constants}
\begin{itemize}
\item[(i)] \textbf{Homogeneous eq.}:\\
\vspace{0.3cm}
\item[(ii)]\textbf{Inhomogeneous eq.}:\\
\vspace{6cm}
\end{itemize}
\end{frame}


\begin{frame}\frametitle{``Homogeneous" differential equation}
Diff. eq. of the form
\begin{block}{}
\vspace{0.5cm}
\end{block}
\vspace{3cm}
\end{frame}


\section{2.4. Existence and uniqueness of solutions}

\begin{frame}\frametitle{Problem}
\underline{Consider our general Cauchy problem}:\\
\vspace{0.2cm}
$\Omega \subset \R \times \R^n$ open,  $(x_0,y_0) \in \Omega$ initial value and $F: \Omega \rightarrow \R^n$ cont. fct.\\
\vspace{0.2cm}
\underline{Aim}: We are interested in finding a solution of the Cauchy problem
\begin{block}{}
\vspace{1cm}
\end{block}
\underline{Questions}:
\begin{itemize}
\item[(a)] Has (CP) always a solution (at least in a small neighb. of $x_0$ in $\R$)?
\item[(b)] Is this solution unique?
\end{itemize}
\vspace{0.2cm}
\underline{Remark}: By Prop.~1 the answer to \textcolor{blue}{(a)} and \textcolor{blue}{(b)} also gives corresp. answers for DEs of higher order.
\end{frame}

\begin{frame}\frametitle{Example}
Consider the following Cauchy problem:\\
\begin{block}{}
\vspace{1.5cm}
\end{block}
\begin{itemize}
\item \underline{First}: $y(x)=0\; \forall x\in \R$ is a solution of $(*)$.\\
\item Ignore for a moment the initial conditions and look for general solutions with \underline{$y(x) \neq 0$} for $x$ in a certain time interval.
\end{itemize}
\vspace{4cm}
\end{frame}

\begin{frame}\frametitle{Peano theorem}
\underline{Moral}: For general $F$,  solutions of $(CP)$ may not be unique. However, concerning \textcolor{blue}{(a)} there is the following general theorem:

\begin{block}{Theorem 1 (Peano)}
Let $F: I \times U \rightarrow \R^n$ be a continuous fct. and $(x_0,y_0) \in I \times U = \Omega$ be initial cond.\\
\textbf{Then}, there exists $\epsilon >0$ and a solution
$$\;$$
of the Cauchy problem $(CP)$.
\end{block}
\vspace{0.2cm}
\underline{Remarks:}
\begin{itemize}
\item There is no uniqueness assertation.
\item 
\end{itemize}
\end{frame}


\begin{frame}\frametitle{Preparation for the Picard-Lindelhöf theorem}
\textbf{Now} consider $F:I \times U \rightarrow \R^n$ cont. fct. and $(x_0,y_0) \in I \times U$.
\begin{block}{}
\vspace{0.5cm}
\end{block}
\underline{Aim}: Establish Picard-Lindelhöf theorem which guarantees the existence and uniqueness of the solutions of the Cauchy problem above. \\
We need the following lemma:
\begin{block}{Lemma 1}
Let $F: \Omega \rightarrow \R^n$ be a continuous fct. and $(x_0,y_0) \in \Omega$ be initial cond.\\
A continuous fct $y:I \rightarrow U \subset \R^n$ is a solution of the Cauchy problem $(CP)$ $\iff$ $\forall x\in I$
$$\vspace{0.2cm}$$
\end{block}
\end{frame}

\begin{frame}\frametitle{Preparation for the Picard-Lindelhöf theorem}
\underline{Hence}: we need to \textbf{find} a complete metric space of fcts  s.t. the map $T$\\
\vspace{0.5cm}
is a contracting map from this space to itself. This means that we can apply Banach's fixed point theorem.\\
\vspace{0.2cm}
\underline{We need a Lipschitz character for the function $F$}:\\
\begin{block}{}
\vspace{2cm}
\end{block}
\begin{block}{Corollary 1}
Let $F:I \times U \rightarrow \R^n$ cont.  diff.  fct w.r.t. $y$-variable (i.e. all $F_x$ are diff.). 
Assume further that $U$ is convex and that $d_2F:I \times U \rightarrow \mathcal{L}(\R^n,\R^n)$ is bounded.\\
\textbf{Then}, $F$ satisfies $(L_{I,U})$.
\end{block}
\end{frame}

\begin{frame}\frametitle{Picard-Lindelhöf theorem}
\begin{block}{Theorem 2 (Picard-Lindelhöf,  Version A)}
Let $F: \Omega \rightarrow \R^n$ be a cont. fct satisfying $(L_{I,U})$ (w.r.t. second variable) and let $(x_0,y_0) \in \Omega=I \times U.$\\
\textbf{Then}, there exist $\epsilon >0$ s.t. $I_\epsilon:=(x_0-\epsilon, x_0+\epsilon) \subseteq I$ and a unique solution
$$\;$$
\vspace{0.2cm}
\textbf{Moreover}, for every $\epsilon' \leq \epsilon$ ($\epsilon' >0$), the restriction
$$\;$$
is the unique solution of $(CP)$ defined on $I_\epsilon$.
\end{block}
\vspace{0.2cm}
\underline{Note}:
Since the proof is based on Banach's fixed point theorem, we can extract an iterative method to construct the unique solution of $(CP)$. 
\end{frame}

\begin{frame}\frametitle{Improved error estimates for the Picard-Lindelhöf iteration}
\underline{Define} inductively\\
\vspace{0.5cm}
Then $y(x):=\lim_{k\rightarrow \infty} y_k(x)$ exists for $x\in I_\epsilon$ for $\epsilon>0$ and is a solution of $(CP)$.\\
This procedure is called the \textcolor{blue}{P.-L. iteration method} of successive approx.:

\begin{block}{Propostion 2 (Improved error estimates for the P.-L. iteration)}
Consider $F: I \times U \rightarrow \R^n$ a cont. fct satisfying $(L_{I,U})$.
Let $A \subset U$ be closed and $(x_0,y_0) \in I \times A$.\\
We set $y_0(x):=y_0$ and inductively $\forall x \in I$ satisfying the following cond.:\\
\vspace{0.8cm}
Let $I_0 \subset I$ be open s.t. $x_0\in I$ and $(A_{k,x})$ holds $\forall x \in I_0, k \in \N$. \textbf{Then}, we have $\forall x \in I_0$:
\begin{itemize}
\item[(a)] \;\\
\vspace{0.4cm}
\item[(b)]\;\\
\vspace{0.4cm}
\end{itemize}
\textbf{Moreover}: $y_0:I_0 \rightarrow A \subset \R^n$ is cont. diff. and solves $(CP)$.
\end{block}
\end{frame}

\section{2.5. Some facts about linear ODEs}

\begin{frame}\frametitle{Some facts about linear ODEs}
\underline{Next} we review some facts about:
\vspace{0.2cm}
\begin{itemize}
\item[\ding{228}] Local and global solutions
\vspace{0.2cm}
\item[\ding{228}] Dependence of solutions on initial values and parameters
\vspace{0.2cm}
\item[\ding{228}] Autonomous DE and vector fields
\end{itemize}
\end{frame}

\begin{frame}\frametitle{Local solutions}

\end{frame}

\begin{frame}\frametitle{Maximal solutions}

\end{frame}

\begin{frame}\frametitle{Dependence on a parameter}
First-order DE depending on a parameter is a DE of the form
\begin{block}{}
\vspace{0.3cm}
\end{block}
where $F:\Omega \times M \rightarrow \R^n$ cont., $\Omega\subset \R \times \R^n$ open and $(M,d)$ metric spcase.\\
\vspace{0.2cm}
\underline{Solution}: It is a fct $y:I \times M \rightarrow \R^n$ s.t. for every $\eta_0 \in M$:
\begin{block}{}
\vspace{0.3cm}
\end{block}
\vspace{0.3cm}
\underline{Example}: $n=1, \Omega=\R \times (0, \infty), M=(0, \infty)$.\\
\vspace{2cm}

\end{frame}

\begin{frame}\frametitle{Dependence on a parameter}
\begin{block}{Theorem 3 (P.-L.  Version B)}
Consider $I \subset \R$ open interval, $U \subset \R^n$ open, $(M,d)$-metric space.\\
Let $F:I \times U \times M \rightarrow \R^n$ cont. fct s.t. $F(\cdot, \cdot, \eta): I \times U \rightarrow \R^n$ satisfies $(L_{I,U})\; \forall \eta \in M$ with Lipschitz const. indep. of $\eta$,  and $(x_0,y_0,\eta_0) \in I \times U \times M$ the initial condition.\\
\textbf{Then}, $\exists \epsilon >0$ s.t. $\forall z \in U$ with $||z-y_0|| < \epsilon$ and $\eta \in M$ with $d(\eta, \eta_0) < \epsilon$ there exists a unique solution $y_{z,\eta}:I_\epsilon \rightarrow U$ of\\
$$\;$$
\textbf{Moreover}, the fct $y:I_\epsilon \times B(y_0,\epsilon) \times B_d(\eta_0,\epsilon) \rightarrow U$ is given by
$$\;$$%y(x,z,\eta):=y_{z,\eta}(x) \text{ is continuous}.$$
\end{block}
\end{frame}


\begin{frame}\frametitle{Dependence of solutions on initial values}
\underline{Aim}: Under certain assumps.,  we want to establish diff. dependence on initial values.\\
\vspace{0.2cm}
\textbf{Consider}
\begin{itemize}
\item[\ding{182}] $F:I \times U \rightarrow \R^n$ cont. and cont. diff. w.r.t. $y$-variable, i.e.  \\
Under this cond.,  $F$ satisfies local Lipschitz cond. on $I \times U$.\\
Fix $x_0\in I, U_0 \subset U$ open and consider the family of initial value problem:\\
\vspace{1.3cm}
\item[\ding{183}] Assume that we have a family of solutions:\\
\vspace{1.5cm}
\end{itemize}
\underline{Objective}: Any family of solutions \textcolor{blue}{\ding{182}}:\\
\vspace{0.2cm}
%$$\;$$%y:I \times U_0 \rightarrow U \text{ is cont. diff.}$$
\underline{Known}: It is cont. diff. w.r.t. $x$ (by def. of a solution).\\
\vspace{0.2cm}
\underline{Problem}: Existence and continuity of $d_2 y$.
\end{frame}

\begin{frame}\frametitle{Dependence of solutions on initial values}
\begin{block}{Proposition 3 (dep. of initial value)}
Let $I \subset \R$ be an open interval s.t.  $x_0\in I$ and $U_0 \subset U \subset \R^n$ be open subsets.\\
Assume that $F: I \times U \rightarrow \R^n$ satisfies \textcolor{blue}{\ding{182}} and let $y:I \times U_0\rightarrow \R^n$ be a cont.  family of solutions of \textcolor{blue}{\ding{183}}.\\
\textbf{Then},  $d_2y: I \times U_0 \rightarrow \mathcal{L}(\R^n, \R^n)$ exists and \\
\vspace{1cm}
In particular, $d_2 y$ is continuous.

\end{block}

\begin{block}{Theorem 4 (generalization of the Prop.~3)}
Let $I \subset \R$ be an open interval s.t.  $x_0\in I$ and $U_0 \subset U \subset \R^n$ be open subsets.\\
Let $F: I \times U \rightarrow \R^n$ be of class $\mathcal{C}^k, k \in N \cup \{\infty\}$ and 
$y:I \times U_0 \rightarrow \R^n$ be a family of solutions of
$$\;$$
Then $y$ is of class $\mathcal{C}^k$.
\end{block}
\end{frame}

\begin{frame}\frametitle{Autonomous DE and vector fields}
\begin{itemize}
\item \textbf{Autonomous DE}: \\
e.g.
\vspace{0.5cm}
\item \textbf{Vector fields}: A (cont., diff., $\mathcal{C}^k$-) vector field on $U$ is a (cont., diff., $\mathcal{C}^k$-) function $F:U \rightarrow \R^n$.\\
\begin{itemize}
\item[(a)] $F(x)=v \in \R^n, \forall x\in U$ is a constant vector field.\\
\vspace{0.2cm}
\item[(b)] grad$(f)=\Big(\frac{\partial f}{\partial x_1},\frac{\partial f}{\partial x_2},\cdots, \frac{\partial f}{\partial x_n}\Big), $ where $f:U \rightarrow \R$ is cont. diff.\\
\vspace{0.2cm}
\item[(c)] $y'=F(y)$, where $F:U \rightarrow \R^n$ is a cont.  vector field on $U$.
\begin{eqnarray*}
&\Rightarrow& \text{ find the so-called \textbf{integral curves} of the v.f.}\\
&\Rightarrow& \text{ find a family of curves (parametrized by i.v.) s.t. each vector} F(z) \\
&& \text{ of the v.f. is tangent at } z \text{ to the curve of the family passing through } z\\
&\Rightarrow& y'(x) \text{ is the tangent vector to the curve} x\mapsto y(x) \text{ at } z=y(x) \text{ and}\\
&& \text{ has to be equal to } F(z)=F(y(x))
\end{eqnarray*}

\item[(d)] \underline{Derivative}: $X: U \rightarrow \R^n$ cont. v.f. ($U \subset \R^n$ open)\\
\vspace{2.5cm}
\end{itemize}
\end{itemize}
\end{frame}


\begin{frame}\frametitle{Conserved quantity for an autonomous DE}
Consider $U \subset \R^n$ open and $X: U \rightarrow \R^n$ a cont. vector field.\\
Let 
\begin{block}{}
\vspace{0.5cm}
\end{block}
and $g:U \rightarrow \R$ a fct of class $\mathcal{C}^1$.\\
$\Rightarrow$ $g$ is called \textbf{conserved quantity} of $(*)$ (or of $X$) if for all solutions $y:I \rightarrow U$ of $(*)$ the function
$$\;$$ %I \ni x \mapsto g(y(x)) \text{ is continuous}.$$

\begin{block}{Properties}
\begin{itemize}
\item[(a)] If $Xg(y)=0 \;\forall y \in U \Rightarrow g$ is a conserved quantity of $X$.
\item[(b)] Assume that $\forall y_0 \in U\; \exists$ a solution $y: I \rightarrow U$ of $(*)$ and $x_0 \in I$ s.t. $y(x_0)=y_0$ (this is always the case due Peano and P.-L. theorems).\\
\textbf{Then},  every conserved quantity of $X$ satisfies $Xg(y)=0\; \forall y \in U$.
\end{itemize}
\end{block}
\end{frame}

\begin{frame}\frametitle{Examples}
\begin{itemize}
\item[(i)] Special case of Newton's law:\\
\vspace{3cm}
\item[(ii)] \; \vspace{2cm}
\end{itemize}
\end{frame}

\begin{frame}\frametitle{Autonomous DE and vector fields}
\underline{Uniqueness and existence:}
Let $U \subset \R^n$ be open and\\
\vspace{0.2cm}
$\phi: \R \times U \rightarrow U$ be a \textbf{one parameter transformation group}, i.e. \\
\vspace{5cm}
Note that by using Thm.~4: \begin{itemize}
\item $\phi$ is of class $\mathcal{C}^k$
\item $\phi_t$ is $\mathcal{C}^k$-differentiable.
\end{itemize}
\end{frame}


\begin{frame}\frametitle{Overview of this chapter}

\end{frame}

}


\end{document}