\documentclass{article}
\usepackage{graphicx} % Required for inserting images

\title{Resources for Analysis Refresher Course}
\author{Guendalina Palmirotta}
\date{June 2024}
\usepackage{hyperref}
\hypersetup{
    colorlinks=true,
    linkcolor=blue,
    filecolor=magenta,      
    urlcolor=blue,
    %pdftitle={Overleaf Example},
    pdfpagemode=FullScreen,
    }
\begin{document}

\maketitle


As is usual in mathematics, \textbf{doing exercises},
is the best way to learn and understand material. Students are encouraged to attempt exercises from
the textbooks/lecture notes/exercise sheets referenced below. All the mentioned resources from this document can be found and download in the OneDrive folder:\\
Refresher in Analysis - Literature\footnote{
\url{https://uniluxembourg-my.sharepoint.com/:f:/g/personal/guendalina_palmirotta_uni_lu/Emkxs9_LSLpIrrLetxXE-5IBmMdKkcuXM6M-d6cQsEjQ1A?e=Jor5GW&CT=1718021963646&OR=OWA-NT-Mail&CID=a45cde1c-8199-573b-5f36-5c8c3d312e23}}\\
Password: RefresherAnalysis2024.\\

\noindent
\underline{Disclaimer:} The resources below cover much more material than there will be in the refresher
course, and certain parts of the material below are not relevant to courses in the Masters programme
at Luxembourg. Hence, to find the relevant parts of each of the materials below, at least for the
refresher course, it is best to look at the contents of the refresher course listed below.

%\tableofcontents



\section*{Analysis refresher course contents}
\begin{itemize}
    \item Basic real analysis: functions of one variable and several variables, derivatives, inverse and implicit function theorems
    \item Ordinary Differential Equations: elementary solution techniques like separation of variables and variation of the constant, existence and uniqueness theorem (Picard/Lindelöf), some facts about linear ODE
    \item Basic complex analysis (of one variable): definition of holomorphic, power series development, identity theorem, Cauchy Riemann equations, complex path integration and use of the residue theorem for the computation of real integrals
\end{itemize}
\newpage
\section*{Textbooks}
\begin{itemize}
    \item \textit{Advanced Calculus}, A Geometric View, James J. Callahan, Springer (2010).
    \item \textit{Complex Analysis}, Third Edition, Undergraduate Texts in Mathematics, Joseph Bak and Donald J. Newman, Springer (2010).
    \item \textit{Mathematical Analysis I}, Second Edition, Claudio Canuto and Anita Tabacco, Springer (2015).
    \item \textit{Visual Complex Analysis}, Tristan Needham, Clarendon Press, Oxford (1997).
    \item \textit{Understanding Analysis}, Second Edition, Undergraduate Texts in Mathematics, Stephen Abbott, Springer (2010).
\end{itemize}

\section*{Notes freely available online}
\begin{itemize}
    \item Analysis 3 (with solutions of the exercises) by Job Kuit\\
    \url{https://math.uni-paderborn.de/ag/rg/team/kuit/analysis-3}
    \item Complex Analysis Lecture Notes by Dan Romik:\\ \url{https://www.math.ucdavis.edu/~romik/data/uploads/notes/complex-analysis}
    \item Complex Analysis with Applications by Jacob Shapiro:\\
\url{https://web.math.princeton.edu/~js129/PDFs/teaching/MAT330_spring_2023/MAT330_Lecture_Notes.pdf}
\item Ordinary Differential Equation by Alexander Grigorian:\\
\url{https://www.math.uni-bielefeld.de/~grigor/odelec2007.pdf}
\end{itemize}

\section*{Online (youtube) courses}
Check Steve Brunton's youtube channel: \url{https://www.youtube.com/@Eigensteve/videos}.
\begin{itemize}
\item Differential Equations and Dynamical Systems by Stve Brunton:\\
\url{https://www.youtube.com/watch?v=9fQkLQZe3u8&list=PLMrJAkhIeNNTYaOnVI3QpH7jgULnAmvPA}
\item Crash Course in Complex Analysis by Steve Brunton:\\
\url{https://www.youtube.com/watch?v=_mv0q7-WF4E&list=PLMrJAkhIeNNQBRslPb7I0yTnES981R8Cg}
\item Complex analysis, A Visual and Interactive Introduction
by Juan Carlos Ponce Campuzano:\\
\url{https://complex-analysis.com/}
\end{itemize}
\end{document}
